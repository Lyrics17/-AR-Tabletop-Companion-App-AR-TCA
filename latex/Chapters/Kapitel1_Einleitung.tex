\setlength{\parindent}{0em}

% Chapter Template

\chapter{Einleitung} % Main chapter title

\label{Kapitel 1} % Change X to a consecutive number; for referencing this chapter elsewhere, use \ref{ChapterX}

%----------------------------------------------------------------------------------------
%	Projektübersicht
%----------------------------------------------------------------------------------------

\section{Projektübersicht}

%-----------------------------------
%	Motivation
%-----------------------------------
\subsection{Motivation}

Ich habe erst vor Kurzem, ca. einem Jahr, angefangen mich für das Warhammer 40k Hobby zu interessieren. Dementsprechend sehe ich mich immer noch als Anfänger, wenn es um den Spielanteil dieses Hobbys geht. Zwar habe ich bereits ein paar Runden gespielt, verfüge aber noch nicht über eine tiefgehende Kenntnis aller Regeln. Jedoch konnte ich in diesen Runden erkennen, dass man als Anfänger nahezu erschlagen wird an Information über Einheiten, einzelne Phasen im Spiel, besondere Aktionen etc. Die meisten dieser Daten stehen in den Kodexen der jeweiligen Fraktion. Dabei werden diese Kodexe ausschließlich in Buchform vertrieben. Das bedeutet, dass die ersten dutzend Runden zum großen Teil aus suchen und umblättern in diesen Büchern besteht. Das zieht die Spiele zum einen sehr in die Länge und zum anderen kann es zu Fehlern und Diskussionen unter den Spieler sorgen.\newline

Mit diesem Projekt wollte ich eine Anwendung schaffen, die es Anfängern erlaubt, möglichst schnell die Grundprinzipien des Spiels zu verinnerlichen, in dem sie einen einfachen und schnellen Weg haben, die wichtiges Eigenschaften ihrer Einheiten und der Geländestücke zu sehen ohne dabei im Kodex oder Regelbuch suchen zu müssen. Dabei werden nur die grundlegendsten Regeln berücksichtigt, um den Einstieg möglichst angenehm zu machen.

%-----------------------------------
%	Projektidee
%-----------------------------------
\subsection{Projektidee}
%ist quasi alles was geplant ist
Die grundsätzliche Idee hinter der Anwendung ist es, Anfängern einen leichteren Einstieg in das Tabletop Spielsystem Warhammer 40k zu bieten. Diese können die Anwendung nutzen, um komfortabler die Profilwerte einer Einheit und Charakteristiken von Geländestücke zu betrachten. Dafür war anfangs geplant, dass man einfach nur ein beliebiges Modell einer Einheit oder ein Geländestück mit der Handykamera abfilmt und die jeweiligen Werte auf dem Bildschirm projiziert werden. Nach einer Recherche wurde aber klar, dass dies eine nach Modellierung der gewünschten Objekte erfordern würde. Was aber den Umfang des Projektes sprengen würde. Angesichts dessen wurde auf QR-Codes ausgewichen. Diese soll ein Nutzer scannen können, um so die Informationen angezeigt zu bekommen.\newline

Ein weiterer Aspekt, welcher die Anfangshürde senken soll, ist die Möglichkeit, die Bewegungsdurchmesser von Einheiten auf dem Bildschirm zu projizieren. Ebenso gilt das auf für die Reichweite der Missionszielmarker. Dabei soll die Projektion die Wahrnehmung des Spielfelds über dem Bildschirm erweitern.\newline

Nachdem diese drei Kernfeatures erarbeitet wurden, kamen noch Funktionen wie dem eigenständigen erstellen von QR-Code für Einheiten, dem Definieren einer Armeeliste, das Festlegen von Geländestücken vor Spielbeginn und einer Würfelsimulation mit oder weniger hohen Priorität auf die Anforderungsliste, um das Gesamtbild der Anwendung abzurunden.

%----------------------------------------------------------------------------------------
%	Zeitplan
%----------------------------------------------------------------------------------------
\section{Zeitplan}

TODO: einfügen wenn fertig

%----------------------------------------------------------------------------------------
%	Features
%----------------------------------------------------------------------------------------

\section{Features}

\textbf{Projektion der Profilwerte einer Einheit}\newline
TODO: ausschreiben \newline

\textbf{Projektion der Geländekategorie und Eigenschaft eines Geländestückes}\newline
TODO: ausschreiben \newline

\textbf{Projektion der Reichweite eines Missionszielmarkers}\newline
TODO: ausschreiben \newline

\textbf{Erstellung von QR-Code für hochgeladene Einheiten}\newline
TODO: ausschreiben \newline

\textbf{Festlegung von genutzten Geländestücken vor Spielbeginn}\newline
TODO: ausschreiben \newline

\textbf{Festlegung der eigenen Armeeliste mit optionalen Kontingenten}\newline
TODO: ausschreiben \newline

\textbf{Eine Anleitung, welche währendem Spiels aufgerufen werden kann}\newline
TODO: ausschreiben \newline

\textbf{Würfelsimulation}\newline
TODO: ausschreiben